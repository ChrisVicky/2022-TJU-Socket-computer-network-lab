\chapter{进度总结及项目分工}

\section{本周进度情况}

本周主要完成了对 GET,HEAD,POST 以及出错情况的解析、分类和针对性返回。同时,通过 dynamic\_buffer 对缓冲区以及发出、接收的报文进行封装,缓解了一部分可能出现的缓冲区溢出问题。在读写磁盘时,通过 stat 等函数进行预处理,避免了 segment fault 的发生。最后,创建了 logger.c 为格式化输出日志提供了接口。任务完成情况如表\ref{tab:renwu}所示。

\begin{table}[htbp!]
    \centering
    \begin{tabular}{p{14pt}p{250pt}p{30pt}p{30pt}}
    \hline\centering
    & \multicolumn{1}{c}{本周任务要求}                                        & 完成 &  备注 \\ \hline 
    1 & 完善服务器功能,能按照 RFC 2616 实现 HEAD、GET 和 POST 的持久连接                      & \checkmark    &  无  \\ 
    1.1 & 按照 RFC2616 响应 GET, HEAD 和 POST 方法         & \checkmark      &  无  \\ 
    1.2 & 支持 4 种 HTTP 1.1 出错代码:400, 404,501,505 & \checkmark      &  无  \\ 
    1.3 & 妥善管理接收缓冲区,避免由于客户端请求消息太常导致的缓冲区溢出问题                                 & \checkmark      &   无 \\ 
    2 & 服务器能处理读写磁盘文件时遇到的错误                                      & \checkmark      &  无  \\
    3 & 创建简化的日志记录模块,记录格式化日志                                & \checkmark     &  无  \\  \hline
    \end{tabular}
    \caption{本周进度完成表}\label{tab:renwu}
    \end{table}

\section{人员分工}

人员分工如表\ref{tab:fengong}所示。

\begin{longtable}{p{4em} p{14em}}
    \hline
    人员 & \multicolumn{1}{c}{项目分工} \\
    \midrule
        刘锦帆 & 完成大部分代码工作,以及协议实现部分 \\ \hline        李镇州 & 完成 client 端的处理以及协议设计部分的写作 \\ 
        \hline
    
      \caption{人员分工表}  \label{tab:fengong}
\end{longtable}


