\chapter{协议设计}

\section{协议设计}

\begin{itemize}
    \item 首先要求服务器能连续响应客户端使用同一个TCP连接同时发送多个请求,即可以实现http管线化。
    \item HTTP/1.1中单个TCP连接在同一时刻只能处理一个请求,为了解决这个问题,HTTP/1.1在RFC 2616中规定了Pipelining。因此RFC 2616中规定:一个支持持久连接的客户端可以在一个连接中发送多个请求(不需要等待任意请求的响应)。收到请求的服务器必须按照请求收到的顺序发送响应。
    \item 对于 HTTP 的并发请求,如果服务器认为其中一个请求是错误的,为了避免服务器把它和其他并发的请求全部拒绝了,服务器需要设计可以识别这条错误的请求的下一条请求,使得尽量多的正确并发请求得到满足。
\end{itemize}

\section{HTTP pipeline 设计}

\begin{enumerate}
    \item 什么是HTTP pipelining: http管线化是一项实现了多个http请求但不需要等待响应就能够写进同一个socket的技术,采用管线化的请求会对页面载入时间产生动态的提高,尤其是当通过高延迟的网络,例如通过卫星网络连接;普通情况下通过同一个tcp数据包发送多个http请求,而http管线化向网络上发送更少的tcp数据包,大幅减轻网络负载;只有幂等的请求能够被管线化,例如get和head请求;post请求不应该被管线化;新建立连接的请求因为无法判断源服务器(代理服务器)是否支持http1.1协议,也不应该被管线化处理。所以,仅在重用已经成功建立的持久化连接的情况下,才可以使用管线化。http管线化需要客户端和服务器双方都能够支持,http1.1规定服务器必须支持管线化,但并未提及服务器必须管线化响应信息,但如果客户端选择管线化的通信方式,服务器必须能够支持和受理。
    \item HTTP pipelining的优势:减少cpu和内存占用,减轻网络堵塞,减轻后续请求的延迟。不采用管道化意味着每次请求必须被应答之后,它的连接才能空闲以便发送下一次请求;不采用管道化会导致平均每个连接带来额外的延迟,或者如果服务器不支持http长连接,进行其他的tcp三次握手增加了额外的请求往返,双倍延迟;不需要牺牲当前的tcp连接, 就能够报告错误.一个单用户客户端对于任何一台服务器或者代理服务器都可以维护不多于两个的连接数.在当前由n台服务器组成的网络中, 任意一台代理服务器对另外的服务器或者代理服务器应该维护2*n个连接.这些指南目的在于提升http响应性能,避免网络堵塞。
\end{enumerate}
