\chapter{协议设计}

\section{方法设计规则}
HTTP 1.1三种基本方法的设计规则(persistent connection)和协议头部格式
\subsection*{GET}
客户端请求指定资源信息,服务器返回指定资源。GET 是默认的 HTTP 请求方法,我们用 GET 方法来提交表单数据,然而用 GET 方法提交的表单数据只经过了简单的编码,同时它将作为 URL 的一部分向 Web 服务器发送,因此,如果使用 GET 方法来提交表单数据就存在着安全隐患上。GET 方法提交的数据是作为 URL 请求的一部分所以提交的数据量不能太大。

\subsection*{HEAD}
获得报文头部信息,不返回报文主体内容。用于确认 URI 的有效性及资源更新的日期时间等。HEAD 方法与 GET 类似,但是 HEAD 并不返回消息体。在一个HEAD请求的消息响应中,HTTP投中包含的元信息应该和一个GET请求的响应消息相同。这种方法可以用来获取请求中隐含的元信息,而无需传输实体本身。这个方法经常用来测试超链接的有效性,可用性和最近修改。 一个HEAD请求响应可以被缓存,也就是说,响应中的信息可能用来更新之前缓存的实体。如果当前实体缓存实体阈值不同(可通过Content-Length、Content-MD5、ETag 或Last-Modified 的变化来表明),那么这个缓存被视为过期了。

\subsection*{POST}
将客户端的数据提交到服务器,POST方法是GET方法的一个替代方法,它主要是向Web服务器提交表单数据,尤其是大批量的数据。POST方法克服了GET方法的一些缺点。通过POST方法提交表单数据时,数据不是作为URL请求的一部分而是作为标准数据传送给Web服务器,克服了GET方法中的信息无法保密和数据量太小的缺点。因此,出于安全的考虑以及对用户隐私的尊重,通常表单提交时采用POST方法。


\section{协议头部格式}
客户端向服务器发送一个请求,请求头包含请求的方法、URI、协议版本、以及包 含请求修饰符、客户信息和内容的类似于MIME的消息结构。服务器以一个状态行作为响应,相应的内容包括消息协议的版本,成功或者错误编码加上包含服务器 信息、实体元信息以及可能的实体内容。通常HTTP消息包括客户机向服务器的请求消息和服务器向客户机的响应消息。这两种类型的 消息由一个起始行,一个或者多个头域,一个只是头域结束的空行和可选的消息体组成。HTTP的头域包括通用头,请求头,响应头和实体头四个部分。每个头域 由一个域名,冒号和域值三部分组成。域名是大小写无关的,域值前可以添加任何数量的空格符,头域可以被扩展为多行,在每行开始处,使用至少一个空格 或制表符。

\section{接受缓冲区设计}
mytcp\_sockets\_allocated是整个tcp协议中创建的socket的个数,由mytcp\_prot的成员 sockets\_allocated指向。mytcp\_orphan\_count表示整个tcp协议中待销毁的socket的个数,由mytcp\_prot的成员orphan\_count指向。mysysctl\_tcp\_rmem表示接收缓冲区的大小限制,其缺省值分别是4096bytes,87380bytes,174760bytes。它们可以通过 /proc文件系统,在 /proc/sys/net/ipv4/tcp\_rmem 中进行修改。struct sock的成员 sk\_rcvbuf 表示接收缓冲队列的大小,其初始值取mysysctl\_tcp\_rmem[1],成员sk\_receive\_queue 是接收缓冲队列,结构跟sk\_write\_queue相同。tcp socket的接收缓冲队列的大小既可以通过/proc文件系统进行修改,也可以通过TCP选项操作进行修改。套接字级别上的选项 SO\_RCVBUF可用于获取和修改接收缓冲队列的大小(即strcut sock->sk\_rcvbuf的值)。

\section{日志记录模块设计}
参考了 Apache 的标准,实现规范化输出,便于调试。


